\documentclass{article}

\usepackage[a4paper, total={6in, 10in}]{geometry}
\usepackage[utf8]{vietnam}
\usepackage{cite}
\usepackage{float}
\usepackage{graphicx}
\usepackage{hyperref}

\begin{document}
\title{
      \LARGE{
            \textbf{
                  Xây dựng mô hình Machine Learning \\ dự báo cháy rừng ở các tỉnh Tây Nguyên \\ dựa vào dữ liệu lịch sử thời tiết.
            }
      }
}

\author{
      Nguyễn Đại Kỳ\\
      19521731\\
      \and
      Văn Viết Hiếu Anh\\
      19521225
      \and
      Lê Văn Phước\\

}

\date{\today} % Date for the report
\maketitle % Insert the title, author and date
\begin{center}
      \begin{tabular}{l l}
            Môn học:              & CS114 - Máy học       \\
            Giảng viên hướng dẫn: & Lê Đình Duy           \\
                                  & Phạm Nguyễn Trường An \\
      \end{tabular}
\end{center}

\tableofcontents

\pagebreak

%----------------------------------------------------------------------------------------
%	SECTION 1
%----------------------------------------------------------------------------------------

\section{Tổng quan}
\qquad Bài viết là về quá trình thực nghiệm nghiên cứu các model Machine Learning với mục đích chọn ra mô hình tối ưu để dự đoán mức độ cháy rừng dựa vào dữ liệu thời tiết trong lịch sử của từng địa phương. Với mục đích hổ trợ trong việc dự đoán để phục vụ trong công tác phòng chống cháy rừng ở nước ta. Vì bài viết là ghi chép của quá trình thực nghiệm nên sẽ có nhiều phương pháp được đưa ra sử dụng.

\subsection{Mô tả bài toán}
\qquad Ở nước ta có 3 thảm họa lớn nhất, gây thiệt hại lớn hàng năm về cả người và của. Cùng với lũ lụt và hạn hán, cháy rừng là một thảm họa gây thiệt hại không chỉ về kinh tế mà còn cả con người và hệ sinh thái. Theo thống kê của Cục Kiểm lâm từ năm 1992 đến 2006, trung bình mỗi năm xảy ra 1254 vụ cháy rừng gây thiệt hại khoảng 6646 ha rừng, trong đó có 2854 ha là rừng tự nhiên và 3791 ha là rừng trồng. Bên cạnh việc nâng cao năng lực phòng cháy chữa cháy rừng (PCCCR) cho lực lượng kiểm lâm như đầu tư trang thiết bị, cơ sở vật chất, xây dựng cơ chế điều hành phối hợp và tuyên truyền nâng cao nhận thức trách nhiệm của chủ rừng và người dân, công tác cảnh báo nguy cơ cháy rừng cũng như tổ chức phát hiện sớm và thông báo kịp thời điểm cháy rừng là rất cần thiết.

Từ đầu năm 2007, Cục Kiểm lâm (Bộ Nông nghiệp và Phát triển Nông thôn) đã lắp đặt và vận hành trạm thu ảnh viễn thám MODIS tại Hà Nội với mục đích chính là phát hiện sớm các điểm cháy rừng (hotspots) trên toàn lãnh thổ Việt Nam. Hệ thống trạm thu của TeraScan đã tự động thu nhận, xử lý và sao lưu dữ liệu ảnh MODIS hàng ngày từ 2 vệ tinh TERRA và AQUA với mô-đun Vulcan tự động xử lý và tạo ra dữ liệu các điểm cháy sử dụng thuật toán ATBD-MOD14\cite{website:atbd-mod14}.

Hệ thống này cung cấp dữ liệu về điểm cháy ghi nhận được từ vệ tinh và lưu lại thời gian và tọa độ cháy. Từ khi bắt đầu lắp đặt đến nay hệ thống dữ liệu cháy của cục kiểm lâm được ghi lại được gần 1 triệu điểm cháy. Nhờ lượng dữ liệu này việc xây dựng một hệ thống tự động phân tích mức độ cháy rừng dựa vào các đặc trưng cơ bản của dữ liệu khí tượng thủy văn là hoàn toàn có cơ sở và khả quan.

Input của bài toán là dữ liệu lịch sử thời tiết trong vòng 1 tháng trở lại và từ đó để model đánh giá địa phương đó vào ngày mai có mức độ cháy được đánh giá ở thang nào.

Chưa làm xong đâu, làm phụ đi

\subsection{Mô tả dữ liệu}

Nguồn dữ liệu của nhóm đến từ các website gồm 3 website chính:

\begin{itemize}
      \item firewatchvn.kiemlam.org.vn: là Hệ thống theo dõi cháy rừng trực tuyến thuộc Cục Kiểm Lâm - Tổng cục Lâm Nghiệp
      \item weather.com: là website của The Weather Channel (TWC) - IBM\cite{website:wiki_twc}
      \item worldweatheronline.com
\end{itemize}

Trong 3 nguồn dữ liệu thì chỉ có worldweatheronline.com sử dụng SSR(Server-side render) còn 2 nguồn còn lại đều sử dụng Ajax để truyền dữ liệu qua lại giữa server.

\subsubsection{Weather Data}
\qquad\emph{Việc tìm các nguồn dữ liệu khác về thời tiết ngoài 2 nguồn trên đã được thực hiện song các nguồn này đều có những điểm thiếu rất quan trọng ví dụ như API của website chỉ cung cấp trong 1 năm trở lại hay các website này không cung cấp đủ nhiều địa phương mà chỉ cung cấp dữ liệu ở những thành phố cụ thể.}

Trên trang web của weather.com dữ liệu thời tiết chỉ hiển thị dữ liệu thời tiết trong 2 năm trở lại (tức là 2021 và 2020). Tuy nhiên vì sử dụng Ajax nên sau khi phân tích và ghi lại các request mà website gửi đi cũng như các response nhận về, việc có thể tìm được các cổng API và phương thức giao tiếp với server, từ đó dùng vào việc khai thác dữ liệu tự động trong nhiều năm trước nữa là hoàn toàn có hy vọng.

Về việc tìm nguồn dữ liệu tương tự đã được khai thác trước, nhóm đã từng thử tìm kiếm nhưng để đạt được yêu cầu chi tiết đến từng địa phương với thời gian kéo dài thì không tìm được dữ liệu nào đạt yêu cầu. Ngay cả khi join vào Slack của \href{https://callforcode.org/}{Call For Code} năm nay để xin hỗ trợ vì đề tài này có liên quan đến cuộc thi thì phía ban tổ chức cuộc thi cũng trả lời rằng dữ liệu này không cung cấp cho thí sinh. Ngoài ra nhóm cũng đã thử gửi mail cho Trung tâm Dự báo khí tượng thuỷ văn quốc gia nhưng cũng không nhận được phản hồi. Hiển nhiên, việc tự đi thu thập dữ liệu là tất yếu.

\subsubsection{Fire Data}
\qquad firewatchvn.kiemlam.org.vn là website tạo ra ý tưởng cho nhóm. Website này cung cấp giao diện tra cứu dữ liệu về các điểm cháy vào từng thời gian cụ thể. Tuy nhiên điểm yếu của website này là xây dựng quá nhiều tính năng và sử dụng Ajax nên dùng những công cụ như Beautiful Soup thì không thể thu thập còn nếu dùng những công cụ như Selenium hay Puppeteer thì tốc độ quá chậm (dữ liệu này kéo dài từ 1/1/2008 đến 5/12/2020 nếu tra cứu từng ngày trên 700 quận, huyện, thành phố thuộc tỉnh,... thì sẽ mất rất nhiều thời gian). Điều này bắt buộc nhóm phải phân tích API mà website đã lấy dữ liệu điểm cháy để tăng tốc độ lấy dữ liệu. Bởi vì những thông tin như bản đồ của địa điểm lấy dữ liệu là không cần thiết, ta hoàn toàn có thể lấy bản đồ địa hình của cả trái đất chỉ cần dùng tọa độ. Việc lấy những thông tin nặng như bản đồ cần thời gian tải rất lâu nên việc tìm ra API mà website sử dụng cũng là cần thiết.


%----------------------------------------------------------------------------------------
%	SECTION 2
%----------------------------------------------------------------------------------------

\section{Các nghiên cứu trước}


%----------------------------------------------------------------------------------------
%	SECTION 3
%----------------------------------------------------------------------------------------

\section{Xây dựng bộ dữ liệu}

\qquad Chương này mô tả quá trình thu thập dữ liệu. Nếu dữ liệu crawling tự động thì mô tả cách viết crawler, các khó khăn gặp phải và các số liệu liên quan. Nếu dữ liệu thu thập thủ công thì mô tả các tiêu chí đặt ra để thống nhất trong nhóm khi thu thập. Làm sao để đảm bảo bộ dữ liệu thu thập thủ công có thể khớp gần giống với ngữ cảnh ứng dụng của bài toán.

Sau đó mô tả các thông số chi tiết của bộ dữ liệu, kèm theo ví dụ minh họa rõ ràng. Bài toán đặt ra các trường hợp dữ liệu nào là khó xử lý, có bao nhiêu mẫu dữ liệu thuộc trường hợp đó, chụp vài mẫu dữ liệu khó đó vào báo cáo để minh họa.

\subsection{Quá trình thu thập dữ liệu}


\subsubsection{Weather Data}

\qquad Có thể nói việc lấy dữ liệu thời tiết là công đoạn gây ra nhiều khó khăn nhất. Đa phần dữ liệu lịch sử thời tiết là rất lớn và các công ty hay tập đoàn công nghệ đều dùng để bán chứ không public trên website của họ. Ngay cả trên giao diện chính của weather.com của IBM cũng chỉ hiển thị dữ liệu thời tiết trong 2 năm trở lại (tức là 2021 và 2020). Tuy nhiên sau khi phân tích và tìm ra API và thử thay đổi các thông số, ta hoàn toàn có thể nhận được dữ liệu trong thời gian xa hơn.

\paragraph{weather.com}
Như đã nói ở trên, sau khi phân tích các request mà website này gửi về server của họ, chúng em đã tìm ra cổng API cung cấp dữ liệu thời tiết. Sau khi thay đổi các thông số, cổng API này chấp nhận cung cấp dữ liệu đến 1/1/2014.

Dữ liệu từ website này cung cấp có độ chính xác đến từng tọa độ, có nghĩa là chỉ cần cung cấp tọa độ (làm tròn đến 2 chữ số thập phân) thì server sẽ trả về thời tiết tại điểm đó tùy vào thời gian mà ta muốn. Tuy nhiên điểm yếu của dữ liệu này là chỉ cung cấp các đặc tính cơ bản nhất của thời tiết tại địa điểm đó gồm: nhiệt độ cao nhất và thấp nhất trong ngày, thời gian mặt trời mọc và lặn, lượng mưa (tích lũy trong 7 ngày, trong 1 tháng hoặc chỉ ngày hôm đó).

Sau khoảng nhiều ngày khai thác và xử lý, nhóm đã lấy được dữ liệu của 5 tỉnh Tây Nguyên vào từng xã từng ngày kéo dài từ 1/1/2014 đến 8/6/2021. Dữ liệu gồm các trường cơ bản sau:

\begin{itemize}
      \item ward: Mã của xã, phường, thị trấn, thị xã. Vì sẽ có những địa điểm trùng tên nên nhóm sử dụng mã để phân biệt các địa phương (mã này cung cấp bởi API của firewatchvn.kiemlam.org.vn\cite{website:firewatch_administrative})
      \item date: là ngày mà record được ghi lại.
      \item max/min: là nhiệt độ cao nhất và thấp nhất được ghi nhận trong ngày(celcius).
      \item sunrise/sunset: là thời gian mặt trời mọc và lặn.
      \item 7\_rain: lượng mưa tổng tính từ ngày chủ nhật gần nhất trước đó (cm)
      \item m\_rain: lượng mưa tổng tính từ ngày 1 của tháng đó(cm)
      \item 24\_rain: lượng mưa ghi nhận trong ngày(cm)
\end{itemize}

Dưới đây là một đoạn mẫu trong dữ liệu.

\begin{table}[H]
      \begin{tabular}{lllllllll}
            ward    & date       & max & min & sunrise  & sunset   & 7\_rain & m\_rain & 24\_rain \\
            24727.0 & 2016-10-17 & 30  & 25  & 05:36 AM & 05:28 PM & 5.94    & 30.91   & 1.82     \\
            24727.0 & 2016-10-18 & 30  & 25  & 05:36 AM & 05:28 PM & 7.54    & 32.66   & 1.75     \\
            24727.0 & 2016-10-19 & 30  & 25  & 05:36 AM & 05:27 PM & 7.54    & 32.96   & 0.3      \\
            24761.0 & 2019-10-14 & 32  & 24  & 05:36 AM & 05:30 PM & 0.02    & 0.07    & 0.0      \\
            24761.0 & 2019-10-15 & 32  & 25  & 05:36 AM & 05:30 PM & 0.02    & 0.07    & 0.0      \\
            24761.0 & 2019-10-16 & 32  & 25  & 05:36 AM & 05:29 PM & 0.02    & 0.07    & 0.02     \\
            24761.0 & 2019-10-17 & 32  & 25  & 05:36 AM & 05:29 PM & 0.02    & 0.07    & 0.0      \\
            24761.0 & 2019-10-18 & 32  & 25  & 05:36 AM & 05:28 PM & 0.02    & 0.07    & 0.0      \\
            24761.0 & 2019-10-19 & 32  & 26  & 05:36 AM & 05:28 PM & 0.02    & 0.07    & 0.0      \\
            24761.0 & 2019-10-20 & 32  & 26  & 05:36 AM & 05:27 PM & 0.02    & 0.1     & 0.0      \\
            24761.0 & 2019-10-21 & 33  & 25  & 05:36 AM & 05:27 PM & 0.02    & 0.1     & 0.0      \\
            24761.0 & 2019-10-22 & 32  & 25  & 05:37 AM & 05:26 PM & 0.02    & 0.1     & 0.0      \\
            24761.0 & 2019-10-23 & 33  & 25  & 05:37 AM & 05:26 PM & 0.0     & 0.1     & 0.0      \\
            24761.0 & 2019-10-24 & 33  & 25  & 05:37 AM & 05:25 PM & 0.0     & 0.1     & 0.0
      \end{tabular}
\end{table}

\paragraph{worldweatheronline.com}
Trong các we

\subsection{Xây dựng bộ dữ liệu}


\subsubsection{Imputation of Data}
\subsubsection{Creation of Dataset}


%----------------------------------------------------------------------------------------
%	SECTION 4
%----------------------------------------------------------------------------------------

\section{Training và đánh giá modelods}

\subsection{Convolutional neural network}
\subsection{Fully-connected neural network}


%----------------------------------------------------------------------------------------
%	SECTION 5
%----------------------------------------------------------------------------------------

\section{Ứng dụng và hướng phát triển}


%----------------------------------------------------------------------------------------
%	BIBLIOGRAPHY
%----------------------------------------------------------------------------------------

\bibliographystyle{plain}
\bibliography{source}

%----------------------------------------------------------------------------------------

\end{document}